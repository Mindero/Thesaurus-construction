\documentclass[coursework]{SCWorks}
% Тип обучения (одно из значений):
%    bachelor   - бакалавриат (по умолчанию)
%    spec       - специальность
%    master     - магистратура
% Форма обучения (одно из значений):
%    och        - очное (по умолчанию)
%    zaoch      - заочное
% Тип работы (одно из значений):
%    coursework - курсовая работа (по умолчанию)
%    referat    - реферат
%  * otchet     - универсальный отчет
%  * nir%ournal - журнал НИР
%  * digital    - итоговая работа для цифровой кафдры
%    diploma    - дипломная работа
%    pract      - отчет о научно-исследовательской работе
%    autoref    - автореферат выпускной работы
%    assignment - задание на выпускную квалификационную работу
%    review     - отзыв руководителя
%    critique   - рецензия на выпускную работу
% Включение шрифта
%    times      - включение шрифта Times New Roman (если установлен)
%                 по умолчанию выключен
\usepackage{preamble}
\begin{document}

% Кафедра (в родительном падеже)
\chair{математической кибернетики и компьютерных наук}

% Тема работы
\title{Построение и сравнение тезаурусов на основе анализа текстовых данных новостных источников}

% Курс
\course{3}

% Группа
\group{351}

% Факультет (в родительном падеже) (по умолчанию "факультета КНиИТ")
% \department{факультета КНиИТ}

% Специальность/направление код - наименование
% \napravlenie{02.03.02 "--- Фундаментальная информатика и информационные технологии}
% \napravlenie{02.03.01 "--- Математическое обеспечение и администрирование информационных систем}
% \napravlenie{09.03.01 "--- Информатика и вычислительная техника}
\napravlenie{09.03.04 "--- Программная инженерия}
% \napravlenie{10.05.01 "--- Компьютерная безопасность}

% Для студентки. Для работы студента следующая команда не нужна.
% \studenttitle{Студентки}

% Фамилия, имя, отчество в родительном падеже
\author{Янченко Вадима Александровича}

% Заведующий кафедрой 
\chtitle{доцент, к.\,ф.-м.\,н.}
\chname{С.\,В.\,Миронов}

% Руководитель ДПП ПП для цифровой кафедры (перекрывает заведующего кафедры)
% \chpretitle{
%     заведующий кафедрой математических основ информатики и олимпиадного\\
%     программирования на базе МАОУ <<Ф"=Т лицей №1>>
% }
% \chtitle{г. Саратов, к.\,ф.-м.\,н., доцент}
% \chname{Кондратова\, Ю.\,Н.}

% Научный руководитель (для реферата преподаватель проверяющий работу)
\satitle{доцент, к.\,ф.-м.\,н.} %должность, степень, звание
\saname{С.\,В.\,Папшев}

% Руководитель практики от организации (руководитель для цифровой кафедры)
\patitle{доцент, к.\,ф.-м.\,н.}
\paname{С.\,В.\,Миронов}

% Руководитель НИР
\nirtitle{доцент, к.\,п.\,н.} % степень, звание
\nirname{В.\,А.\,Векслер}

% Семестр (только для практики, для остальных типов работ не используется)
\term{2}

% Наименование практики (только для практики, для остальных типов работ не
% используется)
\practtype{учебная}

% Продолжительность практики (количество недель) (только для практики, для
% остальных типов работ не используется)
\duration{2}

% Даты начала и окончания практики (только для практики, для остальных типов
% работ не используется)
\practStart{01.07.2022}
\practFinish{13.01.2023}

% Год выполнения отчета
\date{2025}

\maketitle

% Включение нумерации рисунков, формул и таблиц по разделам (по умолчанию -
% нумерация сквозная) (допускается оба вида нумерации)
\secNumbering

\tableofcontents

% Раздел "Обозначения и сокращения". Может отсутствовать в работе
% \abbreviations
% \begin{description}
%     \item ... "--- ...
%     \item ... "--- ...
% \end{description}

% Раздел "Определения". Может отсутствовать в работе
% \definitions

% Раздел "Определения, обозначения и сокращения". Может отсутствовать в работе.
% Если присутствует, то заменяет собой разделы "Обозначения и сокращения" и
% "Определения"
% \defabbr

\intro
В задачах обработки естественного языка (NLP) и информационного поиска (IR) важную роль играет использование различных видов знаний: лексических связей между словами, значений слов, специализированных понятий в предметных областях и знаний. Одним из традиционных способов представления такого рода знаний в системах NLP являются тезаурусы\cite{loukachevitch2021ruthes}.

В контексте компьютерной обработки текста тезаурус представляет собой формализованный ресурс, в котором описаны семантические связи между словами или терминами — например, синонимы, гипонимы и другие виды лексических отношений. Благодаря такой структуре, тезаурусы могут использоваться в автоматизированных системах для анализа, поиска и интерпретации текстовой информации.

Использование готовых тезаурусов, обученных на огромном количестве произведений, не всегда даёт лучший результат из"=за особенностей предметной области. Такие ресурсы, как правило, отражают общую лексику языка, но не учитывают контекстуальные особенности и терминологию конкретных тематик.

В связи с этим всё более актуальным становится автоматическое построение тезаурусов на основе анализа конкретных текстов. Такой подход позволяет выявить семантические связи, характерные именно для выбранного корпуса, что способствует более точному и релевантному представлению знаний.

Цель данной курсовой работы "--- построить и сравнить тезаурусы, созданные с помощью методов PMI, Word2Vec, GloVe и BERT, на основе корпуса новостных текстов. Это позволит оценить, насколько различаются семантические представления в зависимости от выбранной модели и насколько точно каждый подход отражает смысловые связи в реальных текстах.

Для достижения поставленной цели необходимо решить следующие задачи:
\begin{itemize}
  \item изучить методы PMI, Word2Vec, GloVe и BERT в построении тезаурусов;
  \item построить тезаурусы с использованием этих методов;
  \item провести сравнительный анализ полученных тезаурусов
\end{itemize}

\section{Понятие тезауруса}
\section{Алгоритмы построения тезауруса}
\subsection{Статистический метод. PMI}
\subsection{Векторные модели}
\subsubsection{Word2vec}
\subsubsection{GloVe}
\subsection{Контекстуальные модели. BERT}
\section{Методы сравнения тезаурусов}

\section{Практическая часть}
\subsection{Сбор данных}
\subsection{Построение тезаурусов}
\subsection{Сравнение тезаурусов}

\conclusion


% Отобразить все источники. Даже те, на которые нет ссылок.
% \nocite{*}
\inputencoding{cp1251}
\bibliographystyle{gost780uv}
\bibliography{thesis}
\inputencoding{utf8}

% Окончание основного документа и начало приложений Каждая последующая секция
% документа будет являться приложением
\appendix

\end{document}
  